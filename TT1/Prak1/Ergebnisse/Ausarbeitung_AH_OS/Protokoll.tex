\documentclass[10pt]{scrartcl}

\usepackage[utf8]{inputenc}
\usepackage{tabularx}
\usepackage{longtable}
\usepackage[ngerman]{babel}
\usepackage[automark]{scrpage2}
\usepackage{amsmath,amssymb,amstext}
%\usepackage{mathtools}
\usepackage[]{color}
\usepackage[]{enumerate}
\usepackage{graphicx}
\usepackage{lastpage}
\usepackage[perpage,para,symbol*]{footmisc}
\usepackage{listings} 
\usepackage[pdfborder={0 0 0},colorlinks=false]{hyperref}
\usepackage[numbers,square]{natbib}
\usepackage{color}
\usepackage{colortbl}
\usepackage[absolute]{textpos}
\usepackage{float}

\lstset{numbers=left, numberstyle=\tiny, numbersep=5pt, breaklines=true, showstringspaces=false} 
\restylefloat{figure}

%changehere
\def\titletext{Lab 1 : MANET}
\def\titletextshort{Praktikum 1}
\author{André Harms, Oliver Steenbuck}

\title{\titletext}

%changehere Datum der Übung
\date{09.11.2011}

\pagestyle{scrheadings}
%changehere
\ihead{TT1, Schmidt}
\ifoot{Generiert am:\\ \today}

\cfoot{Oliver Steenbuck, André Harms}


\ohead[]{\titletextshort}
\ofoot[]{{\thepage} / \pageref{LastPage}}

\setlength{\parindent}{0.0in}
\setlength{\parskip}{0.1in}

\begin{document}
\maketitle

\setcounter{tocdepth}{3}
\tableofcontents

	\listoftables                                 												% 
	\listoffigures   

\section{Einleitung}
 Es wird zuerst der Testsetup beschrieben

\section{Test Setup}
	Die Geräte 

\section{Babel}
	\subsection{Experiment 1}
	Traffic über 3 Hops	
	
	\subsection{Experiment 2}
	Failover	
	
	\subsection{Experiment 3}
	Traffic über 3 Hobs mit TX1


\section{OLRS}
	\subsection{Experiment 1}
	Traffic über 3 Hops	
	
	\subsection{Experiment 2}
	Failover, recovery nach neustart des abgebauten Routers


\section{Anhang}
	\subsection{Router Konfigurationen}
	


\end{document}

