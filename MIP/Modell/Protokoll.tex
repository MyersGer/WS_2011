	\documentclass[10pt]{scrartcl}

\usepackage[utf8]{inputenc}
\usepackage{tabularx}
\usepackage[ngerman]{babel}
\usepackage[automark]{scrpage2}
\usepackage{amsmath,amssymb,amstext}
%\usepackage{mathtools}
\usepackage[]{color}
\usepackage[]{enumerate}
\usepackage{graphicx}
\usepackage{lastpage}
\usepackage[perpage,para,symbol*]{footmisc}
\usepackage{listings} 
\usepackage[pdfborder={0 0 0},colorlinks=false]{hyperref}
\usepackage[numbers,square]{natbib}

\lstset{numbers=left, numberstyle=\tiny, numbersep=5pt, breaklines=true, showstringspaces=false} 

%changehere
\def\titletext{Praktikum 1 : Modellierung von Raumgestalten}
\def\titletextshort{Praktikum 1}
\author{André Harms, Oliver Steenbuck, Armin Steudte, Carsten Nötzel, Dennis Blauhut, Torben Becker}

\title{\titletext}

%changehere Datum der Übung
\date{26.10.2011}

\pagestyle{scrheadings}
%changehere
\ihead{MI, Thiel Clemen}
\ifoot{Generiert am:\\ \today}

\cfoot{Oliver Steenbuck \\ Andre Harms \\  Armin Steudte \\ Carsten Nötzel \\ Dennis Blauhut \\ Torben Becker}

\ohead[]{\titletextshort}
\ofoot[]{{\thepage} / \pageref{LastPage}}

\setlength{\parindent}{0.0in}
\setlength{\parskip}{0.1in}

\begin{document}
\maketitle

\setcounter{tocdepth}{3}
\tableofcontents
\listoffigures
\lstlistoflistings

\section{Einleitung}

\section{Räumliche Darstellung}

      \begin{figure}[htbp]
        \centering
                \includegraphics[scale=0.5]{img/tile_map_campus_pic}
        \caption{Ausschnitt von Campus als \glqq Tilebased Map\grqq{}}
        \label{img:tile_map}
        \end{figure}  
        
        
      \begin{figure}[htbp]
        \centering
                \includegraphics[scale=0.5]{img/google_maps}
        \caption{Ausschnitt Google-Maps}
        \label{img:google_maps}
        \end{figure}  
        
        
        Um andere Aspekte als die räumlichen Gegebenheiten modellieren zu können, ist es möglich ein Layerkonzept zu verwenden. Hierbei werden die gewünschten Eigenschaften auf anderen Schichten modelliert. Diese Schichten lassen sich dann entsprechend verwenden und auswerten.
Sollte sich herausstellen, dass eine 3-dimensionale Modellierung besser geeignet ist, besteht die Möglichkeit, zu einem 3-D Modell zu wechseln. Hier werden zusätliche Eigenschaften nicht mehr in Layer modelliert sondern in parallelen Räumen, sogenannte Spaces.

\section{Entitäten}

\section{Ebenen}



\end{document}

